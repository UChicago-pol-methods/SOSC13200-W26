\documentclass[letterpaper, 12pt, parskip=full,DIV=10]{scrartcl}

\title{SOSC 13200: Social Science Inquiry II. Section 3.}
\subtitle{University of Chicago, Winter 2026.}

\RequirePackage{assets/template_MOW}

\begin{document}
\nobibliography{assets/bib}
\maketitle

\vskip -10 ex

\textbf{Location:} Cobb Hall 110\\
\textbf{Course time:} Mon/Wed 15:00--16:20\\
\textbf{Timeframe:} 01/05/2026--03/14/2026\\
\textbf{Course github:} \url{https://github.com/UChicago-pol-methods/SOSC13200-W26}

\textbf{Instructor:} Molly Offer-Westort; \href{mailto:mollyow@uchicago.edu}{mollyow@uchicago.edu}\\
\textbf{Office hours:} Mon 13:20--14:40 and Tue 14:00--15:20; book at \url{https://calendar.app.google/wmSJFvfyxnDHFK5C9}\\
\textbf{Office:} Pick Hall 328

\paragraph{Course description and objectives.}
In this course, you will learn to approach data thinking like a social scientist.
That means thinking about where your data comes from and how it is measured, and assessing and describing relationships among variables.
Throughout the course, you will be exposed to statistical tests and other quantitative methods used in social science research.
You will also work on statistical software programming and data visualization.

This session of the course focuses on the theme of the ``credibility revolution'' in the social sciences:
a move toward empirical studies built with quality data, rigorous research design, and appropriate strategies for analysis.
The impact of this work was recognized by the 2021 Nobel prize in economics, awarded to David Card, Joshua Angrist, and Guido Imbens.
We will also consider the contributions of carefully designed field experiments, and the work of 2019 economics Nobel prize winners, Abhijit Banerjee, Esther Duflo, and Michael Kremer.

\paragraph{Textbooks.}
\begin{itemize}
\item Probability \& Statistics:
\begin{itemize}
\item  Wackerly, Dennis, William Mendenhall, and Richard L. Scheaffer (2014). \textit{Mathematical Statistics with Applications}. Cengage Learning.
\item For reference, not assigned \\
\url{https://www.stat.berkeley.edu/~stark/SticiGui/index.htm}
\end{itemize}
\item Coding in R:
\begin{itemize}
\item Verzani, John. \textit{simpleR -- Using R for Introductory Statistics}. Online book: \url{https://cran.r-project.org/doc/contrib/Verzani-SimpleR.pdf}
\item Wickham, Hadley, Danielle Navarro, and Thomas Lin Pedersen (2016). \textit{ggplot2: Elegant Graphics for Data Analysis}. Springer-Verlag. Online book: \url{https://ggplot2-book.org/}
\end{itemize}
\end{itemize}

\paragraph{Mathematical proficiency.}
It is not assumed that students have had prior exposure to probability and statistics, and this course will not use calculus or linear algebra.
Familiarity with these topics may be useful for some of the extra reference readings, but this is not required.

\paragraph{Statistical software.}
The statistical software used in this iteration of this class will be R, which we will use along with the RStudio interface.
R is a statistical language and environment for data manipulation, calculation, and visualization; the software for R and RStudio are free to download online.
R can be a very flexible and powerful tool, and it is widely used in statistical and social science research, as well as in some industry settings.
Instructors in other sessions of this course may use Stata, which is also widely used and relevant in particular in the fields of economics and public policy.
In general, basic familiarity with R, Stata, and python can be useful for research in the social sciences.
You are not assumed to have prior experience with R for this class.

\paragraph{Computing.}
You will need access to a computer throughout the quarter with R and RStudio installed.
If you do not have access to a private computer to install this software, University library computers are equipped with RStudio, and you should be able to save your assignments on Box while using these computers.
If you have any issues with access to a computer, please reach out to me and we will find a solution.

\paragraph{Assignments.}
Homework consists of weekly assignments, submitted in R with compiled reports.
Homeworks are graded on a 0--3 scale: 0 (not submitted), 1 (check--), 2 (check), 3 (check+).
Homeworks are worth 24 points total. Participation is worth 13 points. The in-class midterm is worth 18 points, the in-person final is worth 30 points, and the take-home final is worth 15 points.

\begin{center}
\begin{tabular}{l l l}
\textbf{Component} & \textbf{Points} & \textbf{Due date} \\
\hline
Participation & 13 & -- \\
Homework 1 & 3 & Fri, January 9 (11:59pm) \\
Homework 2 & 3 & Fri, January 16 (11:59pm) \\
Homework 3 & 3 & Fri, January 23 (11:59pm) \\
Homework 4 & 3 & Fri, January 30 (11:59pm) \\
Homework 5 & 3 & Fri, February 6 (11:59pm) \\
Homework 6 & 3 & Fri, February 13 (11:59pm) \\
Homework 7 & 3 & Fri, February 20 (11:59pm) \\
Homework 8 & 3 & Fri, February 27 (11:59pm) \\
Midterm (in class) & 18 & Wed, February 4 \\
Final (in person) & 30 & Exam week (TBA) \\
Final (take home) & 15 & Fri, March 6 (11:59pm) \\
\end{tabular}
\end{center}

\paragraph{Accommodations.}
Please reach out to me directly if you would like to request accommodations for the course to better facilitate your learning.
Student Disability Services (\url{https://disabilities.uchicago.edu/}) is also available to provide you resources and support, and may provide approval for specific academic accommodations.
Informing me in a timely manner will help me to ensure accommodations are met and I am able to implement an appropriate assessment of your learning.

\paragraph{Academic honesty.}
All students are expected to be familiar with and follow the University\textquotesingle s academic honesty policies (\url{https://studentmanual.uchicago.edu/academic-policies/academic-honesty-plagiarism/}). 
No citation is required for code generated by AI, but you are responsible for understanding the code and ensuring it works correctly. 
You will be asked to provide input on our class AI policy, and then will be expected to follow it: \url{https://github.com/UChicago-pol-methods/SOSC13200-W26/blob/main/ai-policy.md}. 

\section*{Course outline.}

\subsection*{Week 1: Course introduction.}
\subsubsection*{Class 1.1 (Monday 1/5).}
\begin{itemize}
\item Overview of course objective: how to formalize and test a theory with data and statistical tools.
\item Some motivating examples.
\item Introduction to coding: what the software will be, how you will download it, basics of setting yourself up with good directory hygiene.
\end{itemize}

\subsubsection*{Readings.}
\begin{itemize}
\item \bibentry{tabarrok2021credibility}
\end{itemize}

\subsubsection*{Learning R.}
\begin{itemize}
\item On your own, set up the swirl course on ``R Programming with Email Notification'' \url{https://github.com/UChicago-pol-methods/swirl}
\item Run the following code: \texttt{install.packages("swirl"); library(swirl); install\_course("R Programming E"); swirl()}
\item Complete sections 1 (Basic Building Blocks) and 2 (Workspace and Files).
\end{itemize}

\subsubsection*{Class 1.2 (Wednesday 1/7).}
\begin{itemize}
\item What kinds of questions can we want to ask in social science research?
How do we start formalizing a question that is tractable and falsifiable?
What population is the question in reference to?
\end{itemize}

\subsubsection*{Readings.}
\begin{itemize}
\item \bibentry{holland1986statistics} Skip sections 5, 8.2.
\item \bibentry{king1994designing} Chapter 1.
\end{itemize}

\textbf{Homework 1 due Friday 1/9 at 11:59pm:}
Install and set up statistical software and class working directory on your computer.
Submit a compiled R document in pdf with your name and date, showing the file working directory.

\subsection*{Week 2: Summarizing data numerically and visually.}
\subsubsection*{Class 2.1 (Monday 1/12).}
\begin{itemize}
\item Working with the Card and Krueger dataset, summarizing data numerically.
Summaries of univariate data (sample mean, sample median, quantiles, sample variance, sample standard deviation).
\end{itemize}

\subsubsection*{Readings.}
\begin{itemize}
\item \bibentry{cardkrueger1994minimum}
\item We will come back to this paper again later this quarter to discuss the content and results; for this first pass, we will just focus on the data.
\end{itemize}

\subsubsection*{Learning R.}
\begin{itemize}
\item In swirl, complete sections 3 (Sequences of Numbers), 4 (Vectors), 5 (Missing Values).
\end{itemize}

\subsubsection*{Statistical reference (not required).}
\begin{itemize}
\item Wackerly et al. Chapter 1: What is Statistics?
\end{itemize}

\subsubsection*{Class 2.2 (Wednesday 1/14).}
\begin{itemize}
\item Working with the Card and Krueger dataset, summarizing data visually.
\item Graphical representation of different types of univariate data: histograms, density plots, boxplot, bar charts.
\item Challenges with measurement.
\end{itemize}

\subsubsection*{Readings.}
\begin{itemize}
\item \bibentry{adcockcollier2001measurement}
\end{itemize}

\subsubsection*{Learning R.}
\begin{itemize}
\item ggplot2 book Chapter 2: First steps.
\end{itemize}

\textbf{Homework 2 due Friday 1/16 at 11:59pm:}
Submit a compiled R document in pdf with your name and date.
Load in Card and Krueger dataset, and produce specified summary statistics and a plot.

\subsection*{Week 3: Probability as a model of the world.}
\textit{No class Monday 1/19 (MLK Day).}
\subsubsection*{Class 3.1 (Wednesday 1/21).}
\begin{itemize}
\item How do we think about what it means for something to be random? Why is this mathematical abstraction useful for social science research?
\item Overview of discrete probability.
\item Conditional probability, Bayes Rule.
\item Why understanding conditional probability is essential to understanding social phenomena, and why relying on your intuition can be wrong.
\end{itemize}

\subsubsection*{Readings.}
\begin{itemize}
\item \bibentry{stark2003earthquake}
\item \bibentry{buchanan2007prosecutor} \url{https://statmodeling.stat.columbia.edu/2007/05/18/the_prosecutors/}
\item \bibentry{hill2004sudden} \url{https://onlinelibrary.wiley.com/doi/10.1111/j.1365-3016.2004.00560.x}
\end{itemize}

\subsubsection*{Learning R.}
\begin{itemize}
\item In swirl, complete sections 6 (Subsetting Vectors), 7 (Matrices and Data Frames), 8 (Logic), and 9 (Functions).
\end{itemize}

\subsubsection*{Statistical reference (not required).}
\begin{itemize}
\item Wackerly et al. Chapter 2: Probability; 2.1--2.4, 2.7, 2.10--2.13.
\end{itemize}

\textbf{Homework 3 due Friday 1/23 at 11:59pm:}
Exercises in defining sample space, mapping events to probabilities.
Using \texttt{sample()} to simulate random processes in R.

\subsection*{Week 4: Joint relationships.}
\subsubsection*{Class 4.1 (Monday 1/26).}
\begin{itemize}
\item Using Angrist \& Krueger data to discuss identifying patterns in data, covariance, correlation.
\item Education and earnings are correlated; preview correlation vs. causation with respect to the effect of education on earnings.
\end{itemize}

\subsubsection*{Readings.}
\begin{itemize}
\item \bibentry{angrist1991does}
\end{itemize}

\subsubsection*{Statistical reference (not required).}
\begin{itemize}
\item Wackerly et al. Chapter 3: Discrete Random Variables and Their Probability Distributions; 3.1--3.3.
\item Wackerly et al. Chapter 5: Multivariate probability distributions; 5.1--5.4, 5.7 (ignore continuous distributions).
\end{itemize}

\subsubsection*{Class 4.2 (Wednesday 1/28).}
\begin{itemize}
\item Using Miguel and Kremer data, conditional means, difference in means.
\item Data visualization: scatterplots, plotting conditional expectation function, faceting plots by category.
\end{itemize}

\subsubsection*{Readings.}
\begin{itemize}
\item \bibentry{ba2021role}
\item \bibentry{miguel2004worms}
\item Additional content on the ``Worm Wars'' regarding replication issues with Miguel \& Kremer; you do not need to read or listen to all of this:
\begin{itemize}
\item \bibentry{vox2015wormwars}
\item \url{https://www.vox.com/2015/7/24/9031909/worm-wars-explained}
\item \bibentry{maintenancephase2021wormwars}
\item \url{https://open.spotify.com/episode/7ujpdCPK5hKPgUYDGV87aV?si=nmXeJSqnR0aRlF5S3gMP9g}
\end{itemize}
\end{itemize}

\subsubsection*{Learning R.}
\begin{itemize}
\item Verzani, Chapter 4: Bivariate Data, pp. 32--40.
\end{itemize}

\subsubsection*{Statistical reference (not required).}
\begin{itemize}
\item Huntington-Klein, Nick (2021). \textit{The Effect}. Chapter 5.
\end{itemize}

\textbf{Homework 4 due Friday 1/30 at 11:59pm:}
Data exploration, visualizing joint relationships.

\subsection*{Week 5: Uncertainty and inference.}
\subsubsection*{Class 5.1 (Monday 2/2).}
\begin{itemize}
\item Using Butler and Broockman data, randomization inference, introduction to hypothesis testing, confidence intervals, p-values and statistical significance.
\item Sampling from a larger population; central limit theorem, one- and two-sample t-tests.
\end{itemize}

\subsubsection*{Readings.}
\begin{itemize}
\item \bibentry{butler2011politicians}
\item \bibentry{gerber2012field} Chapter 3: Sampling Distributions, Statistical Inference, and Hypothesis Testing, Sections 3.1--3.5, pp. 51--70.
\item \bibentry{pager2003mark}
\end{itemize}

\subsubsection*{Statistical reference (not required).}
\begin{itemize}
\item Athey, S., \& Imbens, G. (2017). ``The Econometrics of Randomized Experiments.'' In A. V. Banerjee and E. Duflo, editors, \textit{Handbook of Field Experiments}, 1, pp. 73--140. North-Holland.
\item Wackerly et al. Chapter 7: Sampling Distributions and the Central Limit Theorem; 7.1--7.3.
\item Wackerly et al. Chapter 8: Estimation; 8.5--8.6.
\end{itemize}

\subsubsection*{Class 5.2 (Wednesday 2/4).}
\begin{itemize}
\item In-class midterm.
\end{itemize}

\subsubsection*{Readings.}
\begin{itemize}
\item No readings.
\end{itemize}

\textbf{Homework 5 due Friday 2/6 at 11:59pm:}
Re-create the Pager data based on the description in the text.
Formalize in words what hypotheses are being tested.
Conduct the statistical analyses described.
What is the reference population?
Are these results generalizable? Why or why not?

\textbf{Homework 6 due Friday 2/13 at 11:59pm:}
Replication of Butler and Broockman.
What statistical tests are being used?
How do you interpret the regression coefficients?

\subsection*{Week 6: Bivariate regression: best linear predictor.}
\subsubsection*{Class 6.1 (Monday 2/9).}
\begin{itemize}
\item With Butler and Broockman data, regression as the best linear predictor, least squares visually demonstrated.
\item Interpreting regression coefficients.
\end{itemize}

\subsubsection*{Readings.}
\begin{itemize}
\item \bibentry{buenodemesquita2021thinking} Chapter 5: Regression for Describing and Forecasting.
\end{itemize}

\subsubsection*{Class 6.2 (Wednesday 2/11).}
\begin{itemize}
\item Inference for linear regression.
\item Linear regression with experiments and dummy variables; relationship to t-tests.
\end{itemize}

\subsubsection*{Readings.}
\begin{itemize}
\item None; spend this time working on researching data for your projects.
\end{itemize}

\subsubsection*{Statistical reference (not required).}
\begin{itemize}
\item Huntington-Klein, Nick (2021). \textit{The Effect}. Chapter 13.1; 13.2 through 13.2.5 inclusive only.
\item Wackerly et al. Chapter 11: Linear Models and Estimation by Least Squares; 11.1--11.4.
\end{itemize}

\subsection*{Week 7: Multivariate regression: model building.}
\subsubsection*{Class 7.1 (Monday 2/16).}
\begin{itemize}
\item Regression as linearization of conditional expectation function.
\item With Banerjee et al. data, interpreting regression coefficients (and standard errors) in multivariate regression.
\end{itemize}

\subsubsection*{Readings.}
\begin{itemize}
\item \bibentry{banerjee2015miracle}
\end{itemize}

\subsubsection*{Statistical reference (not required).}
\begin{itemize}
\item Wackerly et al. Chapter 11: Linear Models and Estimation by Least Squares; 11.11--11.12.
\end{itemize}

\subsubsection*{Class 7.2 (Wednesday 2/18).}
\begin{itemize}
\item When can we apply causal interpretations to regression coefficients from observational data?
\end{itemize}

\subsubsection*{Readings.}
\begin{itemize}
\item \bibentry{cardkrueger1994minimum}
\item Now we will come back to this paper, and look at their approach to analysis.
\item \bibentry{gerber2014illusion}
\end{itemize}

\textbf{Homework 7 due Friday 2/20 at 11:59pm: Case File 1 (Data + design audit).}
Short memo + reproducible R output on data provenance, unit of analysis, population, missingness, and core descriptive tables/plots.

\subsection*{Week 8: Regression: inference, challenges to inference, and external validity.}
\subsubsection*{Class 8.1 (Monday 2/23).}
\begin{itemize}
\item Pitfalls of p-values and hypothesis testing; what is the alternative?
\end{itemize}

\subsubsection*{Readings.}
\begin{itemize}
\item \bibentry{amrhein2019retire}
\item \bibentry{gerbermalhotra2008}
\end{itemize}

\subsubsection*{Class 8.2 (Wednesday 2/25).}
\begin{itemize}
\item Statistical power, effect sizes, sample size.
\end{itemize}

\subsubsection*{Readings.}
\begin{itemize}
\item \bibentry{coppock2022power}
\item \bibentry{lakens2022samplesize}
\end{itemize}

\textbf{Homework 8 due Friday 2/27 at 11:59pm: Case File 2 (Primary analysis + interpretation).}
Short memo + reproducible R output with bivariate and multivariate regression, an alternate specification, and interpretation with visuals.

\subsection*{Week 9: Moving beyond multiple linear regression: other approaches.}
\subsubsection*{Class 9.1 (Monday 3/2).}
\begin{itemize}
\item Different types of research questions: prediction, heterogeneity.
\item Some alternative tools, and how machine learning is used in the social sciences.
\end{itemize}

\subsubsection*{Readings.}
\begin{itemize}
\item \bibentry{lundberg2020estimand}
\item \bibentry{athey2019machine}
\end{itemize}

\subsubsection*{Class 9.2 (Wednesday 3/4).}
\begin{itemize}
\item Declaring hypotheses in advance / pre-analysis planning.
\end{itemize}

\subsubsection*{Readings.}
\begin{itemize}
\item \bibentry{chen2021preanalysis}
\item \bibentry{monogan2015preregistration}
\end{itemize}

\paragraph{Take-home final (project components, exam-structured).}
The take-home component of the final assessment is a take-home exam that uses either the same capstone dataset with new prompts and a required extension, or a fresh but similar dataset (TBA).
You will submit a PDF report (R/Rmd to PDF) plus source files; the prompt will include data/measurement, descriptives/visualization, modeling/inference, credibility reflection, and a short bridge-to-next-quarter experiment component.

\end{document}
